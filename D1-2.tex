\documentclass[external]{20120615_deliverable_template_ukob}

\usepackage{xspace}
\usepackage{amsmath,amsthm,amssymb,url,color}
\theoremstyle{definition}
\newtheorem{example}{Example}
\usepackage{xcolor}
\usepackage{color, colortbl}
\usepackage{subfigure}
%\usepackage{showkeys}
\usepackage{listings}

\newcommand{\todo}[2]{\textcolor{magenta}{#1: #2}}

\LGtitle{\LiveGovThirtyTitleFonts \textbf{D1.2 - Mobile Sensor App with Mining Functionality} }
%\LGnumber{WP.NR} % e.g 6.7 no letters D or ID

\LGnumber{1.2}

\LGwp{WP1 - Reality Sensing and Mining}

\LGdissemination{PU - Public}

\LGcontractdate{Month 27, March 2014}

\LGactualdate{Month 26, February 2014}

\LGtask{T1.2, T1.3}

\LGtype{Prototype}

%\LGnature{$<$ Report $|$ Prototype $|$ Demonstrator $|$ Other $>$}

% \LGapproved % activate only if approved
%\LGdraft
\LGfinal

\LGversion{1.0}
%\LGstaffmonths{Resources spent on the deliverable}

%\LGdistribution{WP leaders, PMB members, European Commission}

%\LGkeywords{}


%\LGabstract{Put abstract here (separate paragraphs using
%{\tt\char92 par}.)}
\LGabstract{ 

  In this deliverable we present several reality mining methods and an
  improved version of the sensor collection service.  These methods
  allow the extraction of context information form sensor data
  collected on the mobile device. In particular we include a Human
  Activity Recognition component, a Service Line Detection component
  and a Traffic Jam Detection component. These components are
  implemented in a generic way and integrated into the Live+Gov toolkit.  


}

%\LGhistory{01}{2011-10-01}{First draft}{Yiannis Kompatsiaris}
%\LGhistory{02}{2011-10-02}{Modifications}{Sotiris Diplaris}
%\LGhistory{03}{2011-10-03}{Further Modifications \& Corrections}{Sotiris Diplaris}
%\LGhistory{04}{2011-10-05}{Further Modifications \& Corrections}{Sotiris Diplaris}


%------------------------ macro for change portrait to landscape
\usepackage{calc,graphicx,pdflscape,eso-pic} % needed for printing
                                             % headers and footers
\newlength\landscapewidth
\newlength\landscapeheight
\newcommand\landscapepagestyle{ % command which prepare page
                                % for landscape printing,
                                % change to landscape is
                                % achieved by pdflscape
\clearpage \thispagestyle{empty}
\setlength\landscapewidth{247mm}
\setlength\landscapeheight{161mm}
\AddToShipoutPicture*{\AtPageCenter{ % this make "layer" for
                                     % printing the header and footer
\rotatebox[origin=c]{90}{
\hspace*{-2em} % for adjusting position of
               % header and footer
\parbox{\landscapewidth}{\vskip-.57\landscapeheight
%\centerline{\WEKNOWIT \hfill \textbf{\Large \bf{IDx.x - V0} \normalsize }} %header text for Internal Deliverables
                                                                         % notation: IDx.x V{LGversion}
\centerline{\LiveGovLogo \hfill \textbf{\Large \bf{Dx.x - V0} \normalsize }} %header text for External Deliverables
                                                                        % notation: Dx.x V{LGversion}
\vspace{-5pt} \rule{\landscapewidth}{0.4pt} \\
\rule{0pt}{\landscapeheight} \\
%\rule{\landscapewidth}{0.4pt} \\
\centerline{Page \thepage\ }  % footer text
} } } } }

%--------------------------------------------------------------------------------------------------------------%
\usepackage{appendix}
\pretolerance=10000 % prevent overflow lines for $math$ elements

\begin{document}

% add a \LGaddhistory{version}{date}{reason}{revised by} for each new
% version
\begin{LGhistory}
\LGaddhistory{0.1}{2013-10-14}{Outline}{Heinrich Hartmann}
\LGaddhistory{0.2}{2014-01-06}{Added section on topic modeling}{Christoph Kling}
\LGaddhistory{0.3}{2014-01-07}{Revised Outline}{Heinrich Hartmann}
\LGaddhistory{0.4}{2014-01-07}{Alpha Version}{Heinrich Hartmann}
\LGaddhistory{0.5}{2014-01-21}{Added section on Service Line
  Detection}{Christoph Schaefer}
\LGaddhistory{0.6}{2014-01-22}{Added section on Distributed Geo Matching}{Daniel Janke}
\LGaddhistory{0.7}{2014-01-24}{Added description of Sensor Collector}{Heinrich Hartmann}
\LGaddhistory{0.8}{2014-01-31}{Added Traffic Jam Module Description}{Laura Niittyl\"a}
\LGaddhistory{0.9}{2014-02-04}{Added SVM Classifier Description}{Elisavet Chatzilari}
\LGaddhistory{0.10}{2014-02-06}{Revised section on Human Activity Recognition}{Heinrich Hartmann}
\LGaddhistory{0.11}{2014-02-20}{Incorporated feedback from internal
  reviews.}{Heinrich Hartmann}
\LGaddhistory{0.12}{2014-02-20}{Final corrections and revision of figures.}{Heinrich Hartmann}
\LGaddhistory{1.0}{2014-02-25}{Release of final version to the consortium.}{Heinrich Hartmann}
\end{LGhistory}


\newcommand{\LGaddauthorNoPhone}[3]{\hline  #1 &  #2 & %
   \parbox{3em}{E-mail:} \small #3 \\
}

% add a
% \LGaddauthor{Partner}{Name}{Telephone}{Fax}{Email} for each author
\begin{LGauthors}

\LGaddauthor{UKob}{Heinrich Hartmann}{+49 261 287 2759}%
{+49 261 287 100 2759}{\small hartmann@uni-koblenz.de}

\LGaddauthor{UKob}{Christoph Schaefer}{+49 261 287 2786 }%
{+49 261 287 100 2786}{\small chrisschaefer@uni-koblenz.de}

\LGaddauthor{UKob}{Christoph Kling}{+49 261 287 2702}%
{+49 261 287 100 2702}{\small ckling@uni-koblenz.de}

\LGaddauthor{UKob}{Daniel Janke}{+49 261 287 2747}%
{+49 261 287 100 2747}{\small danijank@uni-koblenz.de}

\LGaddauthorNoPhone{MTS}{Laura Niittyl\"a}%
{\small Laura.Niittyla@mattersoft.fi}

\LGaddauthorNoPhone{CERTH}{Elisavet Chatzilari}
{\small ehatzi@iti.gr}

\end{LGauthors}



\begin{LGExecutiveSummary}
\vspace{10pt} 

In this deliverable we present several data mining methods that are
used to extract context information from the reality of citizens.  In
addition we report on the ongoing improvements to our sensor
collection architecture and communication interfaces.

The revised sensor collector provides improvements on recording
performance on low end devices, battery awareness and communication
interfaces. We present a newly designed sensor exchange format that
allows continues streaming of sensor information and implement a
streaming API for real time monitoring of mobile sensors.

We provide a Human Activity Recognition component that has been
trained to detect multiple activities of human ambulation (sitting,
standing, walking, running, cycling, stairs, lying). The classifier is
deployed on a mobile device, integrated into the toolkit architecture
and tested in the mobility field trial.  Furthermore, we have trained
several different machine learning models for this task and evaluated
the results against each other as well as against state of the art
classifiers. Our best classification accuracy remains with 84 \% still
below the reported accuracy of 97\% in the literature. We are
confident that we can improve our method quickly based on the built up
infrastructure.

The Service Line Detection component is capable of detecting the
precise service line used by a citizen based on the GPS sensor
information captured on the mobile device. The detection is performed
in a two-stage process that compares the received coordinates to
real-time GPS positions of public transportation vehicles and in a
second step to interpolated vehicle positions from time-table and
vehicle track data. Also this service is fully integrated into the
Live+Gov SaaS toolkit and has been successfully tested in the first
field trial.

The Traffic Jam Detection component is able to detect irregularities
in public transportation systems that provide real-time data about
vehicles. It has been integrated into the Live+Gov toolkit as part of
the Server Side Mining Service (C9). Although a thorough evaluation of
the detection quality is difficult to perform, a comparison to other
commercially available services shows promising results, with only
minor differences.

The sensor collector as well as other mobile components remain limited
to the Android platform. Despite several efforts, that are explained
in Section \ref{sec:cross_platform}, our prototypes based on a hybrid
cross-platform development framework were not be able to deliver the
intended results. Therefore, as described in D1.1, we have to fall
back on native development of the individual components on multiple
mobile platforms. 

In Section \ref{sec:DGM} we present research results on Distributed Geo
Matching, that allow to link social media posts (most importantly from
Twitter\footnote{\url{http://www.twitter.com}}) to vehicles of public
transportation that they originate from. The approach addresses the
problem of handling the high volume of incoming vehicle position
updates in a distributed computing architecture.

We conclude this deliverable with a report on Geographic Topic
Analysis in Section \ref{sec:GTA}. The presented method is able to
extract latent topics form reports of issues in the urban environment.
These topics take into account information about category of the
reported issue as well as geographic proximity. This new method was
shown to be superior to current state of the art geographic topic
models in the publication \cite{CCK1}.

\end{LGExecutiveSummary}

% add a \LGaddabbreviation{ABBR}{Explanation} for each abbreviation
\begin{LGAbbreviations}
%\LGaddabbreviation{LG}{Live+Gov}

\LGaddabbreviation{\textbf{API}}
{Application Programming Interface}

\LGaddabbreviation{\textbf{GPS}}
{Global Positioning System}

\LGaddabbreviation{\textbf{GSM}}
{Global System for Mobile Communications}

\LGaddabbreviation{\textbf{HTML}}
{HyperText Markup Language}

\LGaddabbreviation{\textbf{HTTP}}
{Hypertext Transfer Protocol}

\LGaddabbreviation{\textbf{JSON}}
{JavaScript Object Notation}

\LGaddabbreviation{\textbf{REST}}
{Representational State Transfer}

\LGaddabbreviation{\textbf{SVM}}
{Support Vector Machine}

\LGaddabbreviation{\textbf{URL}}
{Uniform Resource Locator}

\LGaddabbreviation{\textbf{UUID}}
{Universal Unique Device Identifier}

\LGaddabbreviation{\textbf{WP}}
{Work Package}

\LGaddabbreviation{\textbf{WIFI}}
{Wireless Fidelity (IEEE 802.11), WLAN}

\LGaddabbreviation{\textbf{WLAN}}
{Wireless Local Area Network}

\LGaddabbreviation{\textbf{XML}}
{Extensible Markup Language}

\LGaddabbreviation{~\\}
{~~~}

\end{LGAbbreviations}

% add a \LGaddterm{Term}{Explanation} for each entry of the glossary
%\begin{LGGlossary}
%\LGaddterm{Term}{Definition text here}
%\LGaddterm{Term2}{Definition text here}
%\end{LGGlossary}

\setcounter{tocdepth}{1}

% add this for a table of contents
\LGTOC

\newpage

\chapter{Introduction}
\label{chap:Introduction}

The ambition of the Live+Gov project is to develop applications for
closing the emerging gap between the citizens and the policy makers by
the help of mobile technologies. These applications provide enhanced
communication channels between the authorities and the citizens, which
facilitate increased transparency, direct participation and
opportunities for collaboration (cf. D2.1). 

In order to meet this goals Live+Gov makes use of the advanced
possibilities that are offered by moderns smart-phone devices which
are becoming widely available to the public. In particular, these
devices are equipped with a rich set of sensors, that allow to infer
information about the environment and context the citizen is currently
placed in. In this deliverable we present several data mining
techniques that allow to extract valuable context information from raw
sensor data. For instance, we recognize the current ambulation mode of
the user (e.g. walking or standing) in Section \ref{sec:HAR} and
detect the use of public transportation in Section \ref{sec:SLD}.
These data mining methods are implemented as generic services and
integrated into the Live+Gov toolkit as part of the Mobile Sensor
Mining Component (C15) and the Server Side Mining Service (C8)
(cf. Figure \ref{fig:toolkit}). The extracted context information can
be used to personalize the user experience (cf. WP3) of the use case
applications or analyze the behavior and service needs of the
citizens.

In this deliverable we also provide an improved and extended version
of the Mobile Sensor Collection Component (C14). The improvements
contain in particular enhancements of the communication interfaces. We
provide a new real-time streaming API as well as a revised sensor data
format (ssf) cf. Section \ref{sec:ssf}.

\begin{figure}[ht]
  \centering
  \includegraphics[width = 0.9
  \textwidth]{img/intro/toolkit_architecture.png}
  \label{fig:toolkit}
  \caption{Live+Gov Toolkit Architecture}
\end{figure}


\clearpage
\chapter{Sensor Collection Service}
\label{chap:sc}
%--------------------------------------------------------------------
% \chapter{Sensor Collection Service}
%--------------------------------------------------------------------
\label{chap:SensorCollection}

\section{Mobile Sensor Collector}
\label{sec:SensorColllection}

In this deliverable we present the a revised sensor collection
component. It offers the following improvements over the last version
presented in D1.1:
\begin{itemize}
\item High performance recordings on Low-End devices. During the field
  trial some low-end devices showed difficulties to record the sensor
  samples with sufficient frequency. We streamlined the architecture
  in order to reduced the overhead caused by the processing of the
  samples. For example, we traded the convenience of a SQL database in
  favor of the performance benefits of a flat file solution.  Also, in
  this process we removed dependency on external SDCF
  library\footnote{\url{http://www.sdcf.eu/}}, and reimplemented core
  parts of the component.
\item Improved battery awareness. Power consumption in the sensor
  collection process is largely driven the GPS sensor. In May 2013
  Google released a new location API, that takes advantage of multiple
  location provider and reduces the power consumption significantly.
  The revised component makes use of this new API, and offers a
  fallback in case the service is not supported by the device.
\item Revised export format. We have improved the data exchange
  formats used for publishing samples to other components and storage
  on the central server. In section \ref{sec:ssf} we specify the
  ssf-Sensor Stream Format, which is inspired by the
  \href{http://en.wikipedia.org/wiki/Common\_Log\_Format}{Common Log
    Format} used by many webservers. It allows easy human inspection
  and processing by standard (UNIX) tools like ``grep'' and ``sed''
  while being reasonably memory efficient.
\item Streaming API. The streaming API is a new communication channel
  between the mobile sensor collector and the central server. It
  allows to transfer the incoming sensor data directly to the server
  using the ssf format.
\item Integration into Live+Gov Service Center. The revised sensor
  collector and sensor storage service is compatible with the Live+Gov
  service center. This feature allows central user management as well
  as health monitoring and centralized log aggregation of the services.
\item Seamless Extendability. The revised architecture can be easily
  extended by further stream processing components. This is realized
  by a dispatcher thread that distributes sensor samples to registered
  components over a simple interface (again using ssf). Implemented
  extensions include the Human Activity Recognition component
  (c.f. section \ref{sec:HAR}) and a GPS-sample publisher component
  that is used by the Service Line Detection service (cf. section
  \ref{sec:SLD}).
\end{itemize}

\subsection{Architecture Description}

The new architecture is sketched in Figure \ref{fig:sc_architecture}.
It consists of the following components:

\begin{figure}[h]
\centering
\includegraphics[width=\textwidth]{img/sc/sc_architecture.png}
\caption{Sensor Collection Architecture}\label{fig:sc_architecture}
\end{figure}

\begin{itemize}
\item {\bf Sensor Event Thread}. The sensor event thread registers
  callback listeners for all configured sensors. When sensor events
  occur the received values are serialized in the ssf
  format (cf. \ref{sec:ssf}) and pushed onto the SensorQueue.
\item {\bf SensorQueue}. The sensor queue is a synchronized queue
  object that stores sensor values as strings.
\item {\bf DispatcherThread}. The dispatcher thread reads sensor
  values from the sensor queue and passes them to several services
  that can be registered. Included services are the persistor service
  and the activity recognition component, the streaming service and GPS
  publication pipeline (for use in the Service Line Detection API).
\item {\bf Persistor}. The persistor service is executed by the
  DispatcherThread and appends the received sensor sample into a
  predefined file. Also a variant with zip-file compression is
  included.
\item {\bf TransferThread}. The transfer thread is an independent
  thread that transfers the persisted samples to the Live+Gov servers.
\item {\bf Monitoring Thread}. The monitoring thread polls the
  different components and gathers run-time information like numbers
  of processed samples or transfer state.
\item {\bf Service Controller}. The service controller starts and
  stops all threads, and takes care of proper configuration of all
  services. It listens to intents specified in the API and changes the
  state of the service accordingly.
\end{itemize}

The supported sensors are summarized in Table
\ref{fig:sensor_table}. They are unchanged from D1.1.

\begin{figure}
\centering
\begin{tabular}{|l|p{0.7\linewidth}|}
  \hline
  Sensor & Description \\
  \hline
  GPS                 & Global position in latitude and longitude. 

                        If available, the GPS samples are gathered via
                        Google's new Play Services location
                        provider\footnote{https://developer.android.com/google/play-services/location.html}.
                        
                        If the Play Services are not available, the service falls back to
                        accessing the GPS samples directly.
  \\ \hline
  Accelerometer       & Measures the acceleration force in $m/s^2$
                        that is applied to a device on all three
                        physical axes. 

                        Also the low-pass and high-pass filtered
                        variants 'Linear Acceleration' and 'Gravity' offered by the Google
                        API can be captured.
  \\ 
  Rotation Vector     & Measures the orientation of a device by
                        providing the three elements of the device's
                        rotation vector.  
  \\
  Gyroscope           & Measures a device's rate of rotation in
                        $rad/s$ around each of the three physical
                        axes. 
  \\ 
  Magnetic field      & Measures the ambient geomagnetic field for all
                        three physical axes in $\mu T$. 
  \\ \hline
  WLAN                & A list of all available wireless local area
                        networks in the transmission range. 
  \\
  Bluetooth           & A list of all available bluetooth clients in
                        the transmission range. 
  \\
  GSM                 & Cellular network operator and the radio cell
                        the mobile phone is connected with. 
  \\ \hline
  Google Activity     & Google recently released an Activity
                        Recognition Library. The sensor collector is able to record activities as sensor
                        values as well. 
  \\
  \hline
\end{tabular}
\caption{Supported sensors of the Sensor Collector}
\label{fig:sensor_table}
\end{figure}

\subsection{Intent API}
\label{subsubsec:IntentAPIdescription}

The communication with other Live+Gov toolkit components is
facilitated through the exchange intent
messages\footnote{\url{http://developer.android.com/guide/components/intents-filters.html}}.
The API was slightly improved since deliverable D1.1. We removed the
explicit ``Start/Stop Service'' intents, since the service now shuts
down automatically when no services are requested. The following list
summarizes all provided intent controls. New intents are marked with
an asterisk (*).

\begin{itemize}
\item {\bfseries Start/Stop Recording.} Starts and stops recording of
  sensor samples. Samples are persisted in a local sensor file.
\item {\bfseries Delete Samples.*} Delete all stored samples on the
  device.
\item {\bf Send Annotation.} This intent allows users to annotate
  their recording by a string value that will be recorded by a
  ``tag-sensor''.
\item {\bfseries Enable/Disable Streaming*.} When enabled, all
  collected sensor samples are streamed over to a central server over
  the network. See section \ref{sec:straming} for more details.
\item {\bfseries Transfer samples.}  Controls the sample transfer
  state machine. When enabled and a network connection is available
  sensor samples are transferred in batches to the server backend.
\item {\bfseries Enable/Disable sample broadcast.} When enabled all
  recorded sensor samples are broadcasted in the form of intents into
  the system. This allows other components, e.g. feature extraction
  and context mining, to make use of the recorded sensor data.
\item {\bfseries Request Status Report.} When this intent is received
  a status report is broadcasted to the system. A summary of returned
  information is shown in Figure \ref{tab:StatusIntent}.
\item {\bfseries Set user ID.*} Set the user id to a given value. This
  intent is used to connect the mining results to the user data in the
  Live+Gov service center.
\item {\bfseries Get GPS samples.*} Returns a summary of the latest
  GPS samples gathered by the system in ssf format. It is intended to
  be used by the service line detection service described in Section \ref{sec:SLD}.
\end{itemize}

When a control intent is received by the component, the appropriate
action is taken. After execution has finished a status report intent
is broadcasted to the system which provides information about whether
the requested method call was successful. Further technical
description of the component and the API are included in D4.1.

\begin{figure}[ht]
\centering
\begin{tabular}{|l|l|l|} \hline
   Key       & Type    & Description                                       \\ \hline
   running   & boolean & True if service is running.                       \\
   recording & boolean & True if sensor samples are recorded.              \\
   storage   & boolean & True if samples are stored in the database.       \\
   transfer  & boolean & True if continues transfer of samples is enabled. \\
   broadcast & boolean & True if samples are broadcasted.                  \\ \hline
\end{tabular}
\caption{Structure of status intents}
\label{tab:StatusIntent}
\end{figure}

\subsection*{Component Testing}

The sensor collector comes accompanied with a testing GUI shown in
Figure \ref{fig:sc_gui}, that provides buttons for the implemented
intent controls API. The returned status information is displayed in
the form of plain text logs, button states and spinners.

The GUI can be used to test the functionality of the component and for
the gathering of training data.

In addition we provide automated unit-test cases for the intent API
along with the source code.


\begin{figure}[h]
\centering
\includegraphics[width=0.3 \textwidth]{img/sc/sc_gui.png}
\caption{Sensor collection testing GUI}\label{fig:sc_gui}
\end{figure}

\subsection{Sensor Stream Format (ssf)}\label{sec:ssf}
The sensor stream format ({\it ssf}) which is used for transfering
sensor values from a mobile device (or another sensor source) to the
Live+Gov data-base. The file format is inspired by the
\href{http://en.wikipedia.org/wiki/Common\_Log\_Format}{Common Log
  Format} used by many webservers. It replaces the XML-based format
used in D1.1. We view incoming sensor values as events and record them
as a simple stream of CSV-rows. Each sensor has an individual prefix
but writes into the same file.

The key advantages of this format as opposed to the previously used
XML format are:
\begin{itemize}
\item Human readability. The format is easy to understand by humans.
\item Flexibility. Every line can be interpreted without the context
  of the file. Therefore ssf-files can be arbitrarily sliced,
  concatenated and filtered using UNIX tools like {\it grep, cat, sed,
    awk}.
\item Streaming support. Lines can be individually transferred over tcp
  sockets or messaging systems. Allowing immediate inspection of the
  recorded samples on a remote system.
\end{itemize}

The rows of the ssf-format have the following structure:
\small
\begin{verbatim}
SENSOR_PREFIX,TIME_STAMP, USER_ID, SENSOR_VALUES
\end{verbatim}
\normalsize

\begin{itemize}
\item \texttt{SENSOR PREFIX}. Identifies the sensor producing the
  sample. The following sensor prefixes are supported: \\
  \texttt{GPS} (GPS sensor), \texttt{ACC} (Accelerometer),
  \texttt{LAC} (Linear Acceleration), \texttt{GRA} (Gravity),
  \texttt{GYR} (Gyroscope), \texttt{MAG} (Magnetometer), \texttt{WIFI}
  (Wifi networks), \texttt{BLT} (Bluetooth), \texttt{GSM} (GSM cells),
  \texttt{ACT} (Google Play Services Activity), \texttt{ERR} (Error
  Value), \texttt{TAG} (Annotations entered by the user).
\item \texttt{TIME STAMP} UNIX timestamp in milliseconds
  e.g. \texttt{1377609577214}
\item \texttt{USER ID} ID that identifies records from the same
  user. The default value is the device-id that is provided by
  Android.
\item \texttt{SENSOR VALUES} Sensor values in individual formats. In
  order to adhere to the CSV standard no \texttt{,}-symbol and
  new-lines may be used in this field.
  \begin{itemize}
  \item \texttt{ACC/GYR/MAG/LAC/GRA} x,y,z-values separated by space characters.
  \item \texttt{GPS} lat,lon,alt-values separated by space characters
  \item \texttt{ACT} Activity Name, Confidence separated by space characters.
  \item \texttt{WIFI} List of access-points, where each visible
    accesspoint is separated by a \texttt{;} and written as \\
    {\it Escaped SSID String/Escaped BSSID String/Frequency in MHz/RSSI in dBm}
  \item \texttt{BLT}
    List of devices where each visible device is separated by a \texttt{;}
    and written as \\
    { \it Escaped Address/Device Major Class/Device Class/Bond
      State/Optional Escaped Name/Optional RSSI }
  \item \texttt{GSM}
    The state of the device written as Service State/Roaming State/Manual
    Carrier Selection State/Escaped Carrier Name/Escaped Signal Strength
    followed by \texttt{:}, followed by a possibly empty list of cells,
    where each cell is separated by a \texttt{;} and written as: \\
    {\it Escaped Cell Identity/Cell Type/RSSI in dBm}
  \end{itemize}
\end{itemize}

Example rows may look as follows:
\small
\begin{verbatim}
ACC,1377605748123,5,0.9813749 0.0021324 0.0142523
GPS,1377605748156,5,50.32124 25.2453 136.5335
WIFI,1341244415,wifiUser,"WiFi AP"/"00:12:42"/2412/-45; \\
                  "Another WiFi AP"/"33:13:53"/2437/-56
TAG,1378114981049,anotherUser,"test tag"
BLT,1385988380374,bluetoothUser,"C8:F7:33:B7:B5:B4" \\
                  /computer/computer laptop/bonded/"LAPTOP"/-46
\end{verbatim}
\normalsize

\subsection{Streaming Service}\label{sec:straming}

The revised version of the sensor collector includes a streaming
service, that transfers recorded samples directly to a remote machine.
The streaming service uses the ZeroMQ networking
library\footnote{http://www.zeromq.org} to transfer single ssf lines
to the Live+Gov machine running a streaming server. The streaming
server appends the received samples to a file (zmq\_stream.ssf).

This simple service allows direct monitoring of the recorded samples,
via simple UNIX command line tools. 
\begin{itemize}
\item \texttt{tail -f zmq\_stream.ssf} - prints the recoded samples to the console.
\item \texttt{tail -f zmq\_stream.ssf | grep GPS} - filters out GPS samples.
\item \texttt{tail -f zmq\_stream.ssf | cut --delimiter=',' --field=1 | logtop}\\
  shows the frequencies of the individual incoming sensor values.
\end{itemize}

\section{Sensor Storage Service}

The sensor storage service runs on the Live+Gov servers and stores the
sensor samples collected by the mobile sensor collection component.
The storage component offers a HTTP/REST interface to upload sensor
data.  The uploaded samples are stored in a
PostgreSQL\footnote{http://www.postgresql.org/} database with
PostGIS\footnote{http://postgis.net/} plugin for Geo queries.

The architecture is sketched in \ref{fig:ss_architecture}. It consists
of the following components.
\begin{itemize}
\item {\bf Request Handler.} Java/Tomcat Servlet that provides the
  HTTP interface for sample upload. The received samples are stored in
  the file system for backup in order to have a backup and passed to
  the DB ingester thread. When both operations where successful a
  response code "202 Accepted" is sent back to the client.
\item {\bf DB Ingester.} Stores sensor samples in a database. 
  The samples are provided in ssf format and written to a PostgreSQL
  database with schema described in Figure \ref{fig:db_scheme}.
\item {\bf Inspection tool.} This component allows to inspect the
  stored samples in the database in a convenient way and prepare and
  export them for usage in the data mining task. Figure
  \ref{fig:inspection} shows a screenshot of two views of the front
  end. We are currently expanding this tool to also include
  visualizations of the data mining results.
\end{itemize}

For integration with the Live+Gov service center, we provide an
external service, that probes the upload URL in regular intervals and
sends health checks and uploads log files to the central server as
described in deliverable D4.2.

\begin{figure}
\centering
\includegraphics[width=0.7\textwidth]{img/sc/ss_architecture.png}
\caption{Sensor Storage Architecture}\label{fig:ss_architecture}
\end{figure}

\begin{figure}
{\small
\begin{verbatim}
sensor_gps (trip_id INT, ts BIGINT, lonlat GEOGRAPHY(Point));
sensor_accelerometer (trip_id INT, ts BIGINT, 
     x FLOAT, y FLOAT, z FLOAT);
sensor_linear_acceleration (trip_id INT, ts BIGINT,
     x FLOAT, y FLOAT, z FLOAT);
sensor_gravity (trip_id INT, ts BIGINT, 
     x FLOAT, y FLOAT, z FLOAT);
sensor_tags (trip_id INT, ts BIGINT, tag TEXT);
sensor_google_activity (trip_id INT, ts BIGINT, activity TEXT);
har_annotation (trip_id INT, ts BIGINT, tag TEXT);
\end{verbatim}}
\label{fig:db_scheme}
\caption{Schema of the Sensor Storage DB}
\end{figure}

\begin{figure}[ht]
  \begin{minipage}[b]{0.45\linewidth}
    \includegraphics[width=0.9 \textwidth]{img/sc/inspection_table.png}
    \caption{Meta information of the sensor recordings}\label{fig:minipage1}
  \end{minipage}\quad
  \begin{minipage}[b]{0.45\linewidth}
    \includegraphics[width=0.9 \textwidth]{img/sc/inspection_raw_tab.png} 
    \caption{Raw data inspection view}\label{fig:minipage2}
  \end{minipage}
  \caption{Inspection Web Tool}
  \label{fig:inspection}
\end{figure}

%%% Local Variables:
%%% mode: latex
%%% TeX-master: "../D1-2"
%%% End:


\clearpage
\section{Cross-Platform Development}
\label{sec:cross_platform}

In order to reach as many citizens as possible with our efforts, it is
important to have the mobile components available on multiple
platforms (e.g. Android, iOS). In deliverable D1.1 we have evaluated
different methods for mutiple platform sensor collection and reality
mining. The basic three possible pathways are (cf. D4.1):
\begin{enumerate}
\item {\bf Native development} of all components for all platforms.
\item {\bf Web-based applications} exploit HTML5 and advanced APIs
  offered by mobile browsers to provide services on the mobile phone.
\item {\bf Hybrid development} uses native wrapper frameworks like
  PhoneGap\footnote{\url{http://phonegap.com/}} or
  Titanium\footnote{\url{http://www.appcelerator.com/}} that provide
  access to native platform and hardware features.
\end{enumerate}

The evaluation criteria included necessary sensor sampling
frequencies, native communication facilities and support for
background services. It was found, that the Titanium framework was
is able to fulfill all three requirements. On the basis of this
evaluation we proposed to implement the mobile reality mining 
services using this framework.

In due course we have started to implement prototypes of sensor
processing applications, such as the {\it
  TitaniumSensorCollector}-package which is provided alongside with
this deliverable.  It turned out, however, that it is not possible to
collect sensor data while running in the background with Titanium. Our
prior evaluation covered both requirements, but it was not tested if
it was cover both requirements at the same time. We were very
surprised to learn that those requirements cannot be met in parallel.

For the practical use of the component, the support for sensor
recordings in the background is very important. Indeed, the
classification of human activities is performed while the device is
worn in a pocket, in which case the display is typically switched
off and the application is running in the background.

As stated in D1.1, we propose independent {\bf native implementation}
as fallback solution in this case. Our initial development is focused
on the Android platform. Using a conversion toolkit we have been able
to port some components (e.g. the sensor collector) to Blackberry.
For support of iOS further development is necessary.

It has to be noted, that independent native development is associated
with high additional effort, as the development environments and APIs
are very different from each other. It has to be assessed if the
resources of the project shall be spent on this task, or rather in
improving the quality of the existing Android implementations.

In the view of these difficulties we have taken steps to reduce the
amount of computation performed on the mobile device. In analogy to
the image recognition task described in D3.1, it is possible to deploy
parts of the logic on a central server. Currently our Service Line
Detection component (c.f. section \ref{sec:SLD}) is implemented in
such a way, thereby paving the way for an iOS port of the Mobility
Field Trial Application. A similar solution is planned for the Human
Activity Recognition Component described in section \ref{sec:HAR}.


\clearpage
\chapter{Reality Mining Methods}

In this chapter we describe several reality mining methods that have
been developed in the Live+Gov project.

The first three methods Human Activity Recognition (Section~\ref{sec:HAR}),
Service Line Detection (Section~\ref{sec:SLD}) and Traffic Jam Detection
(Section~\ref{sec:TJD}) are fully implemented and tested in field trials.
The work on Distributed Geo-Matching (Section~\ref{sec:DGM}) and Geographical Topic
Analysis (Section~\ref{sec:GTA}) are of more scientific nature and have already
lead to a publication \cite{CCK1}.

\section{Human Activity Recognition}
\label{sec:HAR}
%%%%%%%%%%%%%%%%%%%%%%%%%%%%%%%%%%%%%%%%%%%%%%%%%%%%%%
%\section{Human Activity Recognition}
%%%%%%%%%%%%%%%%%%%%%%%%%%%%%%%%%%%%%%%%%%%%%%%%%%%%%%
\label{sec:HAR}

In this section we describe the implementation of the human acitivity
recognition ('HAR') component on the mobile device.  The algorithmic
foundation of this data mining task have been described in detail in
deliverable D2.2 in section 3.2 ``Mobile Sensing and activity
recognition''. We include a brief summary here (\ref{sec:har_method})
for the sake of completeness. Subsection \ref{sec:har_component}
contains the descriptions of the implementation. In subsection
\ref{sec:har_eval} we evaluate our method to the on two data sets and
compare it to the state of the art approaches.

Despite several attempts to port the component to other mobile
platforms we have not been able to do so, due to technical
difficulties which where hard to anticipate. As described in
deliverable D1.1 it was planned to implement this component using the
Appcelerator Titanium\footnote{\url{http://www.appcelerator.com/}}
framework. Moreover, several viability checks were presented. In the
first prototypes
\footnote{\url{https://github.com/HeinrichHartmann/LiveGovWP1/tree/master/mobile/TitaniumSensorCollector}}
it turned out, that it is not possible to collect sensor data while
running in the background with Titanium. This feature is clearly
required for an integrated activity recognition library which shall be
used in the field trials.

As stated in D1.1 foresees, we propose independent native
implementation as fallback solution in this case. So far we have a
running implementation for Android and Blackberry. An iPhone port is
planned, once the implementation is more stable.


\subsection{HAR Method Summary}\label{sec:har_method}

The process of activity recognition uses a pipeline of signal
processing and machine learning techniques. It consists of two phases:
the ``training phase'' and the ``integration phase''. 

\begin{figure}[htbp]
\centering
\subfigure[HAR Training Phase]{
\label{fig:har_overview}
\includegraphics[width=0.5 \textwidth]{img/har/classification_overview.png}
}
\subfigure[HAR Integration Phase]{
\label{fig:integrated_har_overview}
\includegraphics[width=0.25 \textwidth]{img/har/integration_overview.png}
}
\caption{Overview Human Activity Recognition}
\end{figure}


In the training phase (cf. Figure \ref{fig:har_overview}) a group of
volunteers is asked to perform the targeted activities for a certain
amount of time, while recording sensor samples with the device in
their pocket.  The gained training data stored in a database and used
to train a classifier of the activities.

In the integration phase (cf. \ref{fig:integrated_har_overview}), the
trained classifier is embedded into the mobile device.

Both phases rely on the preprocessing steps of "windowing'' and
``feature generation''. The stream of incoming sensor data is divided into
time windows of fixed size (typically 1-10 sec.) and for each window
a set of features is computed. This features are filtering out
relevant information from the raw signal. Typical features include
mean values and standard deviations as well as frequency domain
features like Fourier modes.

Deliverable D2.2 contains detailed lists of all sensors and features
and data mining methods that are used in the literature as well as
discussions of quality of the recognition results.

\subsection{Component Description}\label{sec:har_component}

Architecture Description.

\begin{figure}[htbp]
\centering
\includegraphics[width=\textwidth]{img/har/classification_architecture.jpg}
\caption{Classification Architecture}\label{fig:classification_architecture}
\end{figure}

\begin{itemize}
  \item \texttt{SAMPLE PRODUCER} The samples chosen to be used to train the 
    classifier are present in a CSV file. Because of reusability the file has 
    to be imported line by line so it is possible to reuse the whole streaming
    pipeline, which is already in place for the mobile device.
  \item \texttt{WINDOW PRODUCER} Since the classification of the current 
    activity uses a timeframe, the window producer groups them together. It
    allows different window sizes and overlaps.
  \item \texttt{INTERPOLATION \& QUALITY CHECK} To remove frequency disparity 
    and ignore timeframes below a set sample frequency a quality check is in 
    place. After the check, all windows get standardized to a constant frequency
    using interpolation between the data points.
  \item \texttt{FEATURE CALCULATION} The classification runs on different 
    features extracted from the windows (cf. \ref{sec:har_features}).
  \item \texttt{CSV-PERSISTOR} For further processing of the calculated data
    it is required to be store. This is done using a simple CSV data format,
    to ensure cross program importability.
\end{itemize}

\begin{figure}[htbp]
\centering
\includegraphics[width=\textwidth]{img/har/integration.jpg}
\caption{Integrated HAR Architecture}\label{fig:integrated_har}
\end{figure}

\begin{itemize}
  \item \texttt{SENSOR EVENT THREAD} The sensor collector is connected to nearly
    all sensors available on the device. The samples get pushed into a queue to 
    keep the calculation time needed for each sensor event at a minimum.
  \item \texttt{DISPATCHER THREAD} This thread distributes the sensor samples to
    each connected component like the persistor or the human activity 
    recognition.
  \item \texttt{HAR PIPELINE} The pipeline reuses most parts described in 
    \ref{fig:classification_architecture}. The difference here is the fully 
    trained classifier, which tries to identify the current activity.
  \item \texttt{INTENT BROADCASTER} After the classifier recognised an activity,
    it gets broadcasted using an intent.
\end{itemize}


\subsection{Features}\label{sec:har_features}

\begin{itemize}
  \item Mean of each individual axis
  \item Variance of each individual axis
  \item Mean of standard deviation
  \item Variance of standard deviation
  \item Bin distribution of standard deviation
  \item Bin distribution of frequency domain
  \item Device tilting angle
  \item Energy
  \item Kurtosis
\end{itemize}


\subsection{Classifier Training}\label{sec:har_classifier_training}

The first step is separating the data set into a train and test set. Here we 
used data from different users for each activity. After both data sets got 
processed by the classification architecture (cf. Figure \ref{fig:classification_architecture})
the training and test set get imported into {\it Weka}\footnote{\url{http://www.cs.waikato.ac.nz/ml/weka/}}.
Weka comes with build in machine learning algorithms to choose from. Here the 
J48 implementation of the C4.5 algorithm is used to create a decision tree out 
of the provided training set. The test set is used to validate the created tree
and calculate the accuracy and confusion matrix. Since Weka allows to export 
the created tree as Java class, it is possible to integrate the output directly 
into the HAR Pipeline.

% TODO CE
% Describe the component implementation.
% This can be done in a similar fashion like the sensor collector
% further up: Add Bullet point list for the individual components.
% And describe them in one or two sentences.
% * Add a list for the features we use in the current implementation
% * Describe how we train the decision tree classifier using WEKA.
%   Add a screenshot of the results. 
% * Describe how we integrate the classifier into the application

% TODO:
% * Describe Dection Tree Classifier [HH]
% * Describe SVM [CERTH]

\subsubsection*{SVM based classifier}

A popular algorithm to train a binary classification model for mapping
features to activities is the Support Vector Machines (SVMs). SVMs are
known for their ability in smoothly generalizing and coping
efficiently with high-dimensionality pattern recognition
problems. They define a hypothesis space that includes all the
possible linear separations of the data
(Fig.~\ref{fig:svm_hypothesisSpace}) and they choose the one that
maximizes the margin between the two classes
(Fig.~\ref{fig:svm_margin}).

\begin{figure}[h]
\centering
  \includegraphics[width=0.5\textwidth]{img/svms/Svms_hypothesisSpace.pdf}
  \caption{Left: Hypothesis space including all linear separations. Right: The selected hypothesis maximizes the margin.}
  \label{fig:svm_hypothesisSpace}
\end{figure}


\begin{figure}[h]
\centering
  \includegraphics[width=0.5\textwidth]{img/svms/Svm_Maxmargin.pdf}
  \caption{Maximization of the margin}
  \label{fig:svm_margin}
\end{figure}

\noindent The hyperplane that optimally separates the positive and the
negative class (i.e. maximizes the margin) can be described by
$\mathbf{w\cdot x + b = 0}$, where $\mathbf{w}$ is normal to the
hyperplane and $\frac{b}{\parallel\mathbf{w}\parallel}$ is the
perpendicular distance from the hyperplane to the origin
(Fig.~\ref{fig:svm_wb}). The optimal hyperplane can be obtained by
solving the following Quadratic Programming optimization problem:

\begin{equation}\label{Eq:svmQP0}
  \min \frac{1}{2}\parallel\mathbf{w}\parallel \quad \text{s.t.} \quad y_i(\mathbf{w}\cdot x_i + b) -1 \ge 0 \quad \forall i
\end{equation}

In order to relax the constraints of Eq.~\ref{Eq:svmQP0} and allow for
some misclassified points, a slack variable $\xi_i, i=1,\ldots,L$ is
introduced which transforms Eq.~\ref{Eq:svmQP0} into:

\begin{equation}\label{Eq:svmQPxi}
  \min \frac{1}{2}\parallel\mathbf{w}\parallel + C\sum_{i=1}^{L}\xi_i \quad \text{s.t.} \quad y_i(\mathbf{w}\cdot x_i + b) -1 +\xi_i \ge 0 \quad \forall i
\end{equation}

\noindent where the parameter $C$ controls the trade-off between the
slack variable penalty and the size of the margin. In the testing
phase, in order to classify an unseen example $x_t$, its distance to
the hyperplane is calculated using the formula $\mathbf{w}\cdot x_t +
b$. This distance indicates the classifier's confidence that the
unseen example $x_t$ belongs to the examined class.

\begin{figure}[h]
\centering
  \includegraphics[width=0.6\textwidth]{img/svms/Svms_wb.pdf}
  \caption{The resulting hyperplane after training an SVM.}
  \label{fig:svm_wb}
\end{figure}

The previous consider linear separation of the data. However, this is
rarely the case for most real world scenarios and for this reason the
kernel trick has been introduced. Applying the kernel trick to the
cases where the classes are not linearly separable in the input
feature space, we manage to map the features to a higher dimension
(Fig.~\ref{fig:svm_highDim}) where they can be linearly separated. For
example, in the case of an RBF kernel the otherwise linear hyperplane
is transformed to a hypersphere (Fig.~\ref{fig:svm_rbf}). For any
kernel $K(x,y)$, the classification model can be represented by a
vector $\textbf{w}$ (i.e. the model parameters), a bias scalar $b$ and
the support vectors $\mathbf{SV}_j, j=1,\ldots,N_{SV}$. For an unseen
example $x_t$, a confidence score is extracted by computing its
distance to the hyperplane of the model:

\begin{equation}\label{Eq:SVMdecision}
  confidence = \mathbf{w}*\sum_{j=1}^{N_{SV}}{K(\mathbf{SV}_j,\mathbf{x}_t)}+b
\end{equation}

\begin{figure}[h]
\centering
  \includegraphics[width=0.6\textwidth]{img/svms/Svm_DimensionMap.pdf}
  \caption{The data are not separable in the input space by a linear hyperplane. Using the kernel function, the data are mapped into a higher dimensional space where they can be linearly separated.}
  \label{fig:svm_highDim}
\end{figure}

\begin{figure}[h]
\centering
  \includegraphics[width=0.6\textwidth]{img/svms/Svm_RBF.pdf}
  \caption{Non-linear separation of the data using the RBF kernel}
  \label{fig:svm_rbf}
\end{figure}


In our case, an SVM model is trained on the previously extracted
features in order to learn the properties that define the examined
activity. The models are trained using the one versus all (OVA)
technique, i.e. all positive examples of the specific activity versus
all negative examples (i.e. the examples of all other activities). The
distance of a vector from the hyperplane indicates our confidence that
during the analysed window, the user performs the examined
activity. High positive values of this score increase our confidence
that this window belongs to the positive class while high negative
values provide strong confidence that the performed activity is not
the examined one.

% The fact that each model can be represented by a single vector and a
% scalar and that the testing process is essentially a vector
% multiplication, renders SVMs the best solution for a real time
% classification framework. This allows for storing the information
% about all the models in the phone memory, while testing is
% computationally very efficient, making it possible for the image
% classification algorithm to run entirely on a mobile phone.

\subsection{Evaluation}\label{sec:har_eval}

We have evaluated our classifier on two different datasets.

The first dataset was gathered on the University Campus in Koblenz in
December 2013.  It contains a total of around $900K$ samples collected
by $10$ volunteers.  The volunteers were instructed to perform the
activities ''walking'', ''running'', ''stairs'' and ''cycling'' on
predefined routes on the university campus (cf. Figure
\ref{fig:data_collection_handout}). The total time effort per
volunteer was about 20-25minutes and a financial reward was offered as
an incentive. After the recording the samples have been inspected
using our inspection tool and the beginning and ending of the
activities were manually stripped in order to avoid noise from holding
the device in the hand.

%
% [TODO: Publication of Dataset!]
%

The other dataset was obtained from the {\it UCI Machine Learning
  Repository}\footnote{\url{http://archive.ics.uci.edu/ml/datasets/Human+Activity+Recognition+Using+Smartphones}}
and was gathered by Davide Anguita, et. al. \cite{Anguita} in 2012.
It contains around $700K$ collected by 30 volunteers.

The number of samples per activity of both datasets are summarized in
Figure \ref{fig:har_datasets}. Both datasets contain only
accelerometer samples, and have been preprocessed that have been
sampled at a fixed rate of $50Hz$.

\begin{figure}
\centering
\begin{tabular}{|l|r|r|} \hline
Activity  & UCI Dataset & UKOB Dataset \\ \hline
sitting   & 113.728     & 80.951        \\
standing  & 121.984     & 320.737       \\
walking   & 110.208     & 292.024       \\
running   & 0           & 31.916        \\
cycling   & 0           & 436.106       \\
stairs    & 188.800     & 30.086        \\
lying     & 124.416     & 0            \\ \hline \hline
Totals    & 659.136     & 903.156       \\ \hline
\end{tabular}

\caption{Number of accelerometer samples by activity and dataset.}
\label{fig:har_datasets}
\end{figure}

\begin{figure}[htbp]
  \centering
  \includegraphics[width=\textwidth]{img/har/data_collection_handout.png}
  \caption{Instructions for data collection in German
    language}\label{fig:data_collection_handout}
\end{figure}


%
% TODO: Describe Evaluation Results
%


%%% Local Variables:
%%% mode: latex
%%% TeX-master: "../D1-2"
%%% End:


\clearpage
\section{Service Line Detection}
\label{sec:SLD}
%%%%%%%%%%%%%%%%%%%%%%%%%%%%%%%%%%%%
% \section{Service Line Detection} %
%%%%%%%%%%%%%%%%%%%%%%%%%%%%%%%%%%%%
\label{sec:SLD}

One part of the user contextualization is the service line
detection. The aim of this component is to recognize if a citizen uses
public transport within the HSL area in Helsinki. If such a usage is
detected, the right service line id with its direction and the further
itinerary will be determined. Based on this information it is possible
to approach higher level problems like network utilization analysis,
personalized reports about traffic jams \ref{sec:TJD} and automated
connecting train recommendations.

The service line detection is implemented as a server side component
which provides a REST API for answering user queries in real time. For
long term evaluations all API calls are recorded and send to the
Live+Gov Service Center on a daily bases. Due to the seamless
integration into the Live+Gov ecosystem, where each user gets an
universal unique id, all received tracks are personalized and can be
combined with additional information coming from other components. In
a later analysis, the service line detection results can be augmented
with the low level human activity recognition data of the same user to
gain a better insight into the users behavior.

\begin{figure}[ht]
\centering
\subfigure[]{
  \frame{\includegraphics[width=0.3\textwidth,natwidth=556.19,natheight=432]{img/SLD/track.pdf}}
  \label{fig:track}
  \setcounter{subfigure}{1}
} 
\subfigure[]{
  \frame{\includegraphics[width=0.3\textwidth,natwidth=556.19,natheight=432]{img/SLD/realtime.pdf}}
  \label{fig:realtime}
  \setcounter{subfigure}{2}
}
\subfigure[]{
  \frame{\includegraphics[width=0.3\textwidth,natwidth=556.19,natheight=432]{img/SLD/static.pdf}}
  \label{fig:static}
  \setcounter{subfigure}{3}
}
\subfigure[]{
  \frame{\includegraphics[width=0.3\textwidth,natwidth=556.19,natheight=432]{img/SLD/interpolatedCoords.pdf}}
  \label{fig:interpolatedCoords}
  \setcounter{subfigure}{4}
}
\subfigure[]{
  \frame{\includegraphics[width=0.3\textwidth,natwidth=556.19,natheight=432]{img/SLD/interpolatedCoordsWithTimes.pdf}}
  \label{fig:interpolatedCoordsWithTimes}
  \setcounter{subfigure}{5}
} 
\subfigure[]{
  \frame{\includegraphics[width=0.3\textwidth,natwidth=556.19,natheight=432]{img/SLD/combined.pdf}}
  \label{fig:combined}
  \setcounter{subfigure}{6}
}
\caption{
The various input data for the HSL service line detection.
}
\label{fig:SLD}
\end{figure}

Assigning a trajectory to a specific service line requires the access
to suitable background knowledge. In the Helsinki case it is two fold:
First of all, the system is fed with all HSL public transportation
schedules and associated geographic information of stops and travel
paths. By means of this dataset the theoretical position of every
vehicle at any point in time can be calculated. A drawback of only
including static time tables is that actual schedule deviations stay
disregarded. This gap is closed by the HSL Live API, which is the
second data source the service line detection relies on. With this
API, the actual positions of some vehicles can be
determined. Especially trams are tracked in realtime and can be
accessed via this interface.  For the majority of vehicles, such as
buses, live data is not yet available. Lacking a complete realtime
coverage is why still some uncertainty remains in the overall
system. Therefore, the main challenge is to combine both kinds of data
to archive the best possible performance.

Just as the background data, the classification algorithm is also
two-stage. Given an user trajectory containing a small set of GPS
coordinates with timestamps (Figure~\ref{fig:track}), in a first step
the algorithm checks if the latest given timestamp is near the actual
time. If so, the HSL Live API is called in order to find all vehicles
which are currently around the user
(Figure~\ref{fig:realtime}). Service lines found below a certain
distance to the user are scored with a high probability to be the true
line belonging to the given track.

In a second step, each coordinate of the given track is compared to
the static route data (Figure~\ref{fig:static}) obtained from the time
tables and paths.  It should be noted that the public transportation
schedules only contain exact arrival times for each stop, but don't
provide any time data for the pathes between them. Since this data is
too sparse to perform reliable service line detection on it, all paths
are rasterized evenly (Figure~\ref{fig:interpolatedCoords}) and the
corresponding time for each intermediate part is interpolated
(Figure~\ref{fig:interpolatedCoordsWithTimes}). This ensures that no
route section falls through the cracks when executing a simple
distance query in time and space. Figure~\ref{fig:combined} shows the
combined view of all data used for the HSL service line detection.

The result of the algorithm is the one service line which has the
highest probability due to the Live API and/or the smallest distance
to the given set of coordinates. During a testing phase, the optimal
values of all involved parameters like distance thresholds, time and
space raster sizes, and the individual scoring weights have to be
learned. As one outcome of the Helsinki field trial, we expect to find
a good heuristic how to setup the various ingredients of the
algorithm.

\subsection{Implementation details}
As a starting point, HSL provided us with all static service line data like arrival times, stop positions, route meta data, and path files in the well known General Transit Feed Specification (GTFS)\footnote{\url{https://developers.google.com/transit/gtfs/}} format. The original data set had a size of 400 MB and was imported into a PostgreSQL database. All paths and stop positions were stored as native geospatial values using PostGIS, a spatial database extender for PostgreSQL. After the import, we found 587 distinct routes, 7534 stops, and a total number of 206,153 trips in the database. In general, a service line like the bus 934 from Myyrm\"{a}ki to Luhtaanm\"{a}ki, has two directions, travels along its route path and stops at its stops. If a service line runs every ten minutes, each trip has its own trip id and is uniquely identifiable.

While these data are too sparse for distance queries, we interpolate the possible positions of a vehicle along its path to ensure that the distance between two consecutive trip positions is always smaller than ten meter. On the one hand, this simplifies a query like "find all trips in a distance of 20 meters around the user which arrive in plus minus one minute from now" a lot. On the other hand, the interpolation and denormalization inflates the data. Resulting in 324,440,457 trip positions annotated with arrival time and trip id, we enlarged the original data to a size of 38 GB.

This size leads the PostGIS index to its limit and prevents response times below 30 seconds for a service line detection query. Therefore, all data was partitionated horizontally in time dimension. The reason is, that all time tables repeat itself every 7 days and an usual query covers only a very small period of time at a specific weekday. Against this background, we divided the data into 24 parts for every day, which results in 7 times 24 subtables in the database. Finally, this setting archives average response times of less than one second for a whole service line detection query, containing of up to 200 single distance queries.

%%% Local Variables:
%%% mode: latex
%%% TeX-master: "../D1-2"
%%% End:


\clearpage
\section{Traffic Jam Detection}
\label{sec:TJD}
In this section we are presenting a traffic jam detection module, that
is able to detect irregularities in public transportation systems that
provide real-time data about vehicles. It has been integrated into the
Live+Gov toolkit as part of the Server Side Mining Service (C9), and
was successfully used in the first Mobility Field Trial (cf. D5.1,
D5.3) for the Helsinki metropolitan area.

\subsection{Related work}

Traffic fluency monitoring and jam detection is part of traffic
control in all major cities and the real time information about road
network fluency is of interest to all network users and traffic
related authorities. Over the last years, several different methods
for detecting jams have been created. In many cities traffic is
monitored via road side cameras detecting fluency in the most
important channels of the network and sends the image to authorities
and web services. Another increasingly popular method for detecting
jams is by traffic message channels (TMC) that collect information
about road network status from roadside sensors and vehicles and
transmits the information via radio signals to the end user. TMC
information is most commonly used in navigators, both on inbuilt
vehicle navigating systems and separate navigator
devices.\footnote{\url{http://www.tisa.org/technologies/tmc/}} Over
the past few years also different smartphone applications, such as
Waze or Inrix have emerged on the market. In these applications
information gathering is commonly done either with the vehicles
connected to the system or by the users who report their journey
conditions around them. Different applications provide different type
of information, usually including one or several of the following
information types: jams, weather, road work and accidents.  What is
common to all known methods for detecting jams, no service is provided
specifically to the public transport users as most services are
targeted for private vehicle drivers and the authorities. Based on our
surveys no service is being used specifically to public transport nor
uses the delay information of public transport vehicles, which makes
the traffic jam detection module an unique tool.

\subsection{Component Description}

The jam detection module frequently polls vehicle location compared to
the scheduled location from the Helsinki public transport service. The
module analyses the vehicle situation and detects where there are
traffic jams in the certain transportation authority’s transportation
area. The interface returns always the current jam situation, so
client does not need to keep the jams in memory.

The module detects the jams based the tram location, taking into
notice the delay and how fast it increases. Also, the jam is only
detected if there are several vehicles filling the same criteria
within the same area.  

The module provides information about the nearest stop to the detected
jam and also the previous stop from where the affected vehicles have
come from. Based on this information current traffic situation is
visualized on the application.  All of Helsinki city tram network can
be drawn on the map, but only lines that are currently affected by
trams are shown on the map at each time. If no jams are detected in
between two stops, this interval is shown with green colored line. If
a jam is detected between stops, the interval is shown with red
colored line. Image \ref{fig:tjd_screenshot} shows a screenshot of the
application on how the jams are presented in Helsinki city center
during the trial.

\begin{figure}
\centering
\includegraphics[width=0.3 \textwidth]{img/tjd/screenshot.jpg}
\caption{Visualization of the current traffic situation in Helsinki based on Jam detection}\label{fig:tjd_screenshot}
\end{figure}

\subsubsection*{\bf Architecture Description}

The JamDetector module simplified architecture is visualized in the
Figure \ref{fig:tjd_architecture}. The public transportation systems
generally differ in each city or area in multiple ways in the
available data and data formats. The different public transportation
systems are isolated in the module and configurable using the Spring
Framework.

\begin{figure}
\centering
\includegraphics[width=0.5 \textwidth]{img/tjd/architecture.jpg}
\caption{Traffic Jam Module Architecture}\label{fig:tjd_architecture}
\end{figure}

\begin{table}
\centering
\begin{tabular}{|l|l|}
\hline
Address     & 213.157.92.101:80 \\ \hline
Path        & /jamdetector/JamService.svc/jams/{authority} \\ \hline
Protocol    & HTTP GET \\ \hline
Data format & JSON \\ \hline
\end{tabular}
\caption{Jam Detection Module Details}
\end{table}

\subsubsection*{\bf Live+Gov SaaS solution compatibility}

The JamDetector follows Live+Gov SaaS architecture. This means that
all connecting clients are authorized. Also the JamDetector provides
the heart beat (health check) information to the SaaS service and
sends the application log to the service.  

\subsection*{API description}

The API lists the jams detected in the certain public transportation
area in JSON format. The jam object contains information about the
vehicles participating the detected jam and vehicle’s nearest
stop/station. This information can be used to visualize the jam
location for the users.

% \subsection*{Data structure}

% The data structure of the JSON-message is described in figure \ref{fig:tjd_datastructure}.

% \begin{figure}[htbp]
% \centering
% \includegraphics[width=0.5 \textwidth]{img/tjd/data_structure.jpg}
% \caption{JSON data structure}\label{fig:tjd_datastructure}
% \end{figure}

% Example of a given message:

% \begin{verbatim}
% URL /jamdetector/JamService.svc/jams/hsl

% [ { "Id" : 0,
%     "IsJam" : true,
%     "SlowVehicles" : [ { "CumulativeDelay" : 13,
%           "NextStop" : { "Code" : "1301453",
%               "Latitude" : 60.197929999999999,
%               "Longitude" : 24.876580000000001,
%               "Name" : "Laajalahden aukio"
%             },
%           "PreviousStop" : { "Code" : "1301455",
%               "Latitude" : 60.195390000000003,
%               "Longitude" : 24.873429999999999,
%               "Name" : "Tiilimäki"
%             },
%           "Stop" : { "Code" : "1301455",
%               "Latitude" : 60.195390000000003,
%               "Longitude" : 24.873429999999999,
%               "Name" : "Tiilimäki"
%             },
%           "Vehicle" : { "Delay" : 13,
%               "Id" : "RHKL00076",
%               "IsOnStop" : false,
%               "Latitude" : 24.873405999999999,
%               "LineDirection" : 2,
%               "LineId" : "1004",
%               "Longitude" : 60.195535999999997,
%               "NextStopIndex" : 2,
%               "Time" : "20140116-102700",
%               "Timestamp" : "/Date(1389860820413+0200)/"
%             }
%         },
%         { "CumulativeDelay" : 72,
%           "NextStop" : { "Code" : "1240419",
%               "Latitude" : 60.203020000000002,
%               "Longitude" : 24.96584,
%               "Name" : "Kyläsaarenkatu"
%             },
%           "PreviousStop" : { "Code" : "1230410",
%               "Latitude" : 60.204529999999998,
%               "Longitude" : 24.96998,
%               "Name" : "Toukoniitty"
%             },
%           "Stop" : { "Code" : "1230410",
%               "Latitude" : 60.204529999999998,
%               "Longitude" : 24.96998,
%               "Name" : "Toukoniitty"
%             },
%           "Vehicle" : { "Delay" : 82,
%               "Id" : "RHKL00065",
%               "IsOnStop" : false,
%               "Latitude" : 24.968769999999999,
%               "LineDirection" : 2,
%               "LineId" : "1006",
%               "Longitude" : 60.203980999999999,
%               "NextStopIndex" : 3,
%               "Time" : "20140116-102700",
%               "Timestamp" : "/Date(1389860820405+0200)/"
%             }
%         }
%       ],
%     "SlowVehiclesInJamCount" : 2
%   } ]
% ]
% \end{verbatim}

% In this example the JamDetector has detected one jam and there are two
% vehicles in the jam. In addition to the vehicle information the next,
% previous and nearest stop information is returned. The jam location
% can be approximated either from the participating vehicle locations or
% stop locations.  The interface contains more information than the
% current clients require. This is for debugging and possible
% visualization purposes. For example vehicle’s line information is such
% information.

% If there are no jams currently detected, the interface returns:
% \begin{verbatim}
% []
% \end{verbatim}


\subsection{Evaluation}
The module development begun with evaluating different methods for
creating the detection algorithms and creating a way to detect jams
reliably. The tram location data from Helsinki provides reliable
information about the delay of a vehicle in real time and was
considered the most efficient and reliable source of data for the jam
detection.

A comprehensive and reliable detection algorithm was created through
experimenting in the early stages of the module development. The
parameters to be used were searched the most suitable values through
thorough consideration in order to generate reasonable and accurate
alerts that would correspond with the logical situation in traffic and
would fit in the definition of a jam. After some weeks of monitoring
the detections, the defined values were also re-evaluated and
re-defined before the first pilot based on the previous observations. 

Based on the first trial results, the amount of detected jams is seen
relatively high, on average 3 jams were received on each time the API
was called. This would indicate the parameters needing to be set
tighter in the future if only greater jams are wanted to be
detected. When evaluating the accuracy it was quickly discovered that
validating the detection is rather challenging when not having the
possibility to verify the situation on site. User questionnaires
showed that no users were travelling in the areas where alerts were
given at the time of the alert and therefore they could not reliably
state their opinion about the accuracy of alerts. Also, no known major
jams took place during the trial in the pilot area and therefore
manual validation could not be done during the trial. Manual
comparison to other known services providing jam alerts was done. In
all cases the detected jams were of a short duration, mainly less than
a minute and therefore no reliable statement about accuracy could be
done at this stage. In the comparison we used two popular services:
Waze\footnote{\url{http://www.inrixtraffic.com/}}, which is a mobile
application providing traffic data in over 30 countries and
V-traffic\footnote{\url{http://www.v-traffic.fi/}} , a national
service in Finland for providing traffic data in major cities. Based
on the comparison to other services the results seem promising as
alerts were given in the same areas at the same time, only with minor
differences.

When comparing the module to existing services it is obvious that
there are existing systems that provide information with more
geographical coverage and collect the data from greater number of
sources. Thus, these services also provide more alerts as they also
cover roads that are not included in the current module coverage, the
Helsinki tram network. What needs to be addressed when comparing the
module to existing services is that the other services do not focus on
public transport and using only these services would not help to meet
the requirements of the mobility use case in full. However, the
possibility to support the jam detection module with existing services
needs to be further studied. 

We believe that when expanding the data to also cover other public
transport vehicles, the coverage in the module will increase and the
quality and the usefulness of the module improves. For example in the
Helsinki region, all public transport busses, trams, trains and the
ferry will be covered in the new public transport information system
to be implemented in 2016. Then the module will provide a much better
service with greater geographical coverage and expanded
fleet. Unfortunately at the time of development, location data is not
yet available for other vehicles than trams but the possibility to
include these vehicles in the jam detection has been taken into notice
from the start.

As no reliable evaluation is possible at this stage we will continue
validating the module, analyzing the results and developing the
algorithms. This process will continue until the final trial where
improvements and reliable accuracy of the module are hoped to have
been achieved.

%%% Local Variables:
%%% mode: latex
%%% TeX-master: "../D1-2"
%%% End:


\clearpage
\section{Distributed Geo-Matching}
\label{sec:DGM}
\input{sections/DistributedGeoMatching}

\clearpage
\section{Geographic Topic Analysis}
\label{sec:GTA}
\input{sections/TopicModels}

\clearpage
\bibliography{D1-2}

\end{document}
